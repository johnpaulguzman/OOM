% use qualifying logical predicates instead of in relation for quantifiers NOTE DUPLICATE 439
% Operators dump: \odot$ \quad $\oplus$ \quad $\otimes$ \quad $\ominus$ \quad $\oslash
\documentclass[a4paper]{article}

%% Language and font encodings
\usepackage[english]{babel}
\usepackage[utf8x]{inputenc}
\usepackage[T1]{fontenc}

%% Sets page size and margins
\usepackage[a4paper,top=1.25cm,bottom=1.25cm,left=1.0cm,right=1.0cm,marginparwidth=1.25cm]{geometry}

%% Useful packages
\usepackage{amsmath}
\usepackage{graphicx} 
\usepackage[colorinlistoftodos]{todonotes}
\usepackage[colorlinks=true, allcolors=blue]{hyperref}
\DeclareMathSizes{12}{30}{16}{12}
\usepackage[final]{pdfpages}
\usepackage{gensymb}
\usepackage{empheq}
\usepackage[section]{placeins}
\usepackage[most]{tcolorbox}
\usepackage{esint}
\usepackage{amssymb}
\usepackage{centernot}
\usepackage{multicol}
\usepackage{etoolbox}
\usepackage{xparse}
\usepackage{xcolor} % black, blue, brown, cyan, darkgray, gray, green, lightgray, lime, magenta, olive, orange, pink, purple, red, teal, violet, white, yellow.
\usepackage{soul}
\delimitershortfall-1sp
\usepackage{mleftright}
\mleftright


\makeatletter
\newcommand*{\img}[1]{\phantom{\_}%
    \raisebox{-0.5\baselineskip}{%
        \includegraphics[
        height=1.5\baselineskip,
        width=1.5\baselineskip,
        keepaspectratio,
        ]{#1}}\phantom{\_}}
\newcommand{\dfnNEW}[2]{\textcolor{teal}{%
  \hypertarget{#1}{#2}%
    \protected@write\@mainaux{}{%
        \string\expandafter\string\gdef
          \string\csname\string\detokenize{#1}\string\endcsname{#2}}}}
\DeclareDocumentCommand{\dfn}{m o}{\dfnNEW{#1}{\IfNoValueTF{#2}{#1}{#2}}}
\newcommand{\rfrNEW}[1]{\hyperlink{#1}{\csname #1\endcsname}}
\DeclareDocumentCommand{\rfr}{m o o}{%
  \ifcsname#1\endcsname%
    \IfNoValueTF{#3}{\IfNoValueTF{#2}{
      \rfrNEW{#1}}{
      \rfrNEW{#1} #2}}{
      (#2 \rfrNEW{#1} #3)}
  \else%
    \textcolor{red}{\space^!#1}%
  \fi}
\newcommand\rfrlist[1]{\forcsvlist{\rfrlist@item}{#1}}
\newcommand\rfrlist@item[1]{\rfr{#1}\\}
\newcommand{\thmlink}[2]{{}_{\substack{\rfrlist{#1}}}^{\dfn{#2}}}
\newcommand{\TODO}[1]{\colorbox{lime}{\text{#1}}}
\newcommand{\notes}[1]{\underline{\text{#1}}}
\newcommand{\eqComment}[1]{\text{  \# #1}}
\newcommand{\n}{\\[1.5ex] \hline \nonumber \\[0ex]}
\newcommand{\m}{\nonumber \\}


\def\ml{\mleft}
\def\mr{\mright}
\newcommand{\cusand}{,}
\newcommand{\cusor}{`}
\newcommand{\cuspop}[2]{arg_{#1}(#2)}
\newcommand{\cusend}{.}
\newcommand{\cusnum}[2]{{#1}_{{\phantom{ }}_{#2}}}
\newcommand{\free}[1]{{#1}_{free}}
\newcommand{\uset}[1]{\{#1\}}
\newcommand{\otup}[1]{\langle#1\rangle}
\newcommand{\enlist}[2]{{#1}_{1}, {#1}_{2}, {#1}_{3}, \cdots, {#1}_{#2}}

\newenvironment{defAnd}
  {{}_{and}\left\lbrace\begin{aligned}}
  {\end{aligned}\right)}
\newenvironment{defOr}
  {{}_{or}\left\lbrace\begin{aligned}}
  {\end{aligned}\right)}
\makeatother


\title{Next-Next-Gen Notes \\
\large Object-Oriented Maths}
\author{Dark JP}

\begin{document}
\maketitle
\allowdisplaybreaks


\thinmuskip=2mu % commas
\medmuskip=2mu % operators
\thickmuskip=2mu % equalities
\setlength{\belowdisplayskip}{0pt} \setlength{\belowdisplayshortskip}{0pt}
\setlength{\abovedisplayskip}{0pt} \setlength{\abovedisplayshortskip}{0pt}


% https://en.wikipedia.org/wiki/List_of_logic_symbols
Model Theory: semantics; Proof Theory: syntax
\section{Kleene}
\subsection{Linguistic considerations: formulas}
\TODO{undefined terms:} $%
    lolm2k %
$
\begin{tcolorbox}[breakable, enhanced]
\begin{align}
    \n \notes{paradox: logic in terms on logic; solution: compartmentalize logic within "languages"}
    \n \notes{object language/logic: the particular logic to be studied}
    \n \notes{observer's language/logic: the logic used in studying the object language/logic}
    \n \notes{sentences - declarative: a proposition; interrogative: a question; imperative: a command}
    \n \notes{assume that object languages have a class of declarative sentences which serves as the building blocks, -}
    \n \notes{and other sentences can be built from them by certain operations which are called "formulas" (A, ..., O)}
    \n \notes{a language has "prime formulas"/"atoms" (P, ..., Z) which are distinct sentences that don't change meanings}
    \n \notes{a language has 5 operations for building "composite formulas"/"molecules", -}
    \n \notes{and these are - } %
        \dfn{equivalence}[\sim] \notes{: equivalence; } %
        \dfn{implication}[\supset] \notes{: implication; } % 
        \dfn{conjunction}[\&] \notes{: conjunction; } % 
        \dfn{disjunction}[\lor] \notes{: disjunction; } % 
        \dfn{negation}[\lnot] \notes{: negation} % 
    \n \notes{(P, ..., Z) represent distinct prime formulas; (A, ..., O) represent formulas}
    \n \notes{operator precedence: } %
        \rfr{equivalence}, \rfr{implication}, \rfr{conjunction}, \rfr{disjunction}, \rfr{negation}, ..., (\_)%
        \notes{, where the higher ranks are evaluated first, same ranks right first}
    \n \notes{the "scope" of an operator is the parts of the formula where it acts upon}
\end{align}
\end{tcolorbox}


\subsection{Model theory: truth tables, validity}
\TODO{undefined terms:} $%
    lolm2k %
$
\begin{tcolorbox}[breakable, enhanced]
\begin{align}
    \n \notes{this chapter discusses the system of logic called classical logic}
    \n \notes{different systems of logic are conceptually equally possible, but classical logic is the simplest}
    \n \notes{classical logic: assumes that atom/declarative sentence/proposition can either be true or false, but not both}
    \n \TODO{do truth table for: } %
        \rfr{equivalence}, %
        \rfr{implication}, %
        \rfr{conjunction}, %
        \rfr{disjunction}, %
        \rfr{negation}
    \n \notes{"valid"/"identically true"/"tautology" formulas evaluate to true independent of its prime formula values}
\end{align}
\end{tcolorbox}


\subsection{Model theory: the substitution rule, a collection of valid formulas}
\TODO{undefined terms:} $%
    lolm2k %
$
\begin{tcolorbox}[breakable, enhanced]
\begin{align}
    \n \notes{start thm 1, whatever logic done with atoms, you can swap with formulas}
\end{align}
\end{tcolorbox}


% https://en.wikipedia.org/wiki/List_of_mathematical_abbreviations
Note: Operators (op)s preserve type; Relations (rel)s return truths; include setOps; fix
\section{Logic and Set Theory}
\subsection{D: Logical Truths and Operators}
\TODO{undefined terms:} $%
    \dfn{defEq}[:=], %
    \dfn{relEq}[=], %
    (\_), %
    \cusand, %
    \cusor, %
    \cusend, %
$
\begin{tcolorbox}[breakable, enhanced]
\begin{align}
    \n \dfn{truth} [t][] \rfr{defEq}
        \begin{defOr}
                t \rfr{relEq} \dfn{T}
            \\  t \rfr{relEq} \dfn{F}
        \end{defOr}
    \n \dfn{operatorLogic} [\odot][x, y] \rfr{defEq} 
        \begin{defAnd}
                \ml(\rfr{truth} [x][]\mr)
            \\  \ml(\rfr{truth} [y][]\mr)
            \\  \ml(\rfr{truth} [x \odot y][]\mr)
        \end{defAnd}
    \n \dfn{operatorOR} [\dfn{OR}[\lor]][x, y] \cusnum{\rfr{defEq}}{1} \ml(\rfr{truth} [x][]\mr) \cusnum{\cusand}{1} \ml(\rfr{truth} [y][]\mr) \cusnum{\cusand}{1} \ml(\rfr{truth} [x \rfr{OR} y][] \rfr{relEq} \begin{cases} \rfr{F} & x \rfr{relEq} \rfr{F} \cusand y \rfr{relEq} \rfr{F} \\ \rfr{T} & x \rfr{relEq} \rfr{F} \cusand y \rfr{relEq} \rfr{T} \\ \rfr{T} & x \rfr{relEq} \rfr{T} \cusand y \rfr{relEq} \rfr{F} \\ \rfr{T} & x \rfr{relEq} \rfr{T} \cusand y \rfr{relEq} \rfr{T} \end{cases}\mr) \cusnum{\cusend}{1}
    \n \dfn{operatorAND} [\dfn{AND}[\land]][x, y] \cusnum{\rfr{defEq}}{1} \ml(\rfr{truth} [x][]\mr) \cusnum{\cusand}{1} \ml(\rfr{truth} [y][]\mr) \cusnum{\cusand}{1} \ml(\rfr{truth} [x \rfr{AND} y][] \rfr{relEq} \begin{cases} \rfr{F} & x \rfr{relEq} \rfr{F} \cusand y \rfr{relEq} \rfr{F} \\ \rfr{F} & x \rfr{relEq} \rfr{F} \cusand y \rfr{relEq} \rfr{T} \\ \rfr{F} & x \rfr{relEq} \rfr{T} \cusand y \rfr{relEq} \rfr{F} \\ \rfr{T} & x \rfr{relEq} \rfr{T} \cusand y \rfr{relEq} \rfr{T} \end{cases}\mr) \cusnum{\cusend}{1}
    \n \dfn{operatorNOT} [\dfn{NOT}[\lnot]][x] \cusnum{\rfr{defEq}}{1} \ml(\rfr{truth} [x][]\mr) \cusnum{\cusand}{1} \ml(\rfr{truth} [\rfr{NOT} x][] \rfr{relEq} \begin{cases} \rfr{T} & x \rfr{relEq} \rfr{F} \\ \rfr{F} & x \rfr{relEq} \rfr{T} \end{cases}\mr) \cusnum{\cusend}{1} 
    \n \dfn{operatorXOR} [\dfn{XOR}[\veebar]][x, y] \cusnum{\rfr{defEq}}{1} \ml(\rfr{truth} [x][]\mr) \cusnum{\cusand}{1} \ml(\rfr{truth} [y][]\mr) \cusnum{\cusand}{1} \ml(\rfr{truth} [x \rfr{XOR} y][] \rfr{relEq} \begin{cases} \rfr{F} & x \rfr{relEq} \rfr{F} \cusand y \rfr{relEq} \rfr{F} \\ \rfr{T} & x \rfr{relEq} \rfr{F} \cusand y \rfr{relEq} \rfr{T} \\ \rfr{T} & x \rfr{relEq} \rfr{T} \cusand y \rfr{relEq} \rfr{F} \\ \rfr{F} & x \rfr{relEq} \rfr{T} \cusand y \rfr{relEq} \rfr{T} \end{cases}\mr) \cusnum{\cusend}{1}
    \n \dfn{operatorIF} [\dfn{IF}[\implies]][x, y] \cusnum{\rfr{defEq}}{1} \ml(\rfr{truth} [x][]\mr) \cusnum{\cusand}{1} \ml(\rfr{truth} [y][]\mr) \cusnum{\cusand}{1} \ml(\rfr{truth} [x \rfr{IF} y][] \rfr{relEq} \ml(\rfr{NOT} x\mr) \rfr{OR} y \rfr{relEq} \begin{cases} \rfr{T} & x \rfr{relEq} \rfr{F} \cusand y \rfr{relEq} \rfr{F} \\ \rfr{T} & x \rfr{relEq} \rfr{F} \cusand y \rfr{relEq} \rfr{T} \\ \rfr{F} & x \rfr{relEq} \rfr{T} \cusand y \rfr{relEq} \rfr{F} \\ \rfr{T} & x \rfr{relEq} \rfr{T}\cusand y \rfr{relEq} \rfr{T} \end{cases}\mr) \cusnum{\cusend}{1}
%    \n \thmlink{POS-LCmp}{THM-LExp-1} \ml(\rfr{F} \rfr{relEq} x \rfr{AND} \rfr{NOT} x\mr) \cusnum{\rfr{IF}}{1}
%\m \thmlink{THM-LExp-1}{THM-LExp-2} \ml(x\mr) \cusnum{\cusand}{1}
%\m \thmlink{THM-LExp-1}{THM-LExp-3} \ml(\rfr{NOT} x\mr) \cusnum{\cusand}{1}
%\m \thmlink{THM-LExp-2}{THM-LExp-4} \ml(x \rfr{OR} y\mr) \cusnum{\cusand}{1}
%\m \thmlink{THM-LExp-4, THM-LExp-3}{THM-LExp-5} \ml(y\mr) \cusnum{\cusend}{1}
%\m \thmlink{THM-LExp-1, THM-LExp-2, THM-LExp-3, THM-LExp-4, THM-LExp-5}{THM-LExp} \ml(\rfr{F} \rfr{IF} y\mr)
\m \eqComment{a counterexample cannot follow from a false precedence, thus the conditional cannot be false}
    \n \dfn{operatorOIF} [\dfn{OIF}[\impliedby]][x, y] \cusnum{\rfr{defEq}}{1} \ml(\rfr{truth} [x][]\mr) \cusnum{\cusand}{1} \ml(\rfr{truth} [y][]\mr) \cusnum{\cusand}{1} \ml(\rfr{truth} [x \rfr{OIF} y][] \rfr{relEq} \ml(\rfr{NOT} y\mr) \rfr{OR} x \rfr{relEq} \begin{cases} \rfr{T} & x \rfr{relEq} \rfr{F} \cusand y \rfr{relEq} \rfr{F} \\ \rfr{F} & x \rfr{relEq} \rfr{F} \cusand y \rfr{relEq} \rfr{T} \\ \rfr{T} & x \rfr{relEq} \rfr{T} \cusand y \rfr{relEq} \rfr{F} \\ \rfr{T} & x \rfr{relEq} \rfr{T} \cusand y \rfr{relEq} \rfr{T} \end{cases}\mr) \cusnum{\cusend}{1}
    \n \dfn{operatorIIF} [\dfn{IFF}[\iff]][x, y] \cusnum{\rfr{defEq}}{1} \ml(\rfr{truth} [x][]\mr) \cusnum{\cusand}{1} \ml(\rfr{truth} [y][]\mr) \cusnum{\cusand}{1}
\m \ml(\rfr{truth} [x \rfr{IFF} y][] \rfr{relEq} (x \rfr{IF} y) \rfr{AND} (y \rfr{IF} x) \rfr{relEq} \begin{cases} \rfr{T} & x \rfr{relEq} \rfr{F} \cusand y \rfr{relEq} \rfr{F} \\ \rfr{F} & x \rfr{relEq} \rfr{F} \cusand y \rfr{relEq} \rfr{T} \\ \rfr{F} & x \rfr{relEq} \rfr{T} \cusand y \rfr{relEq} \rfr{F} \\ \rfr{T} & x \rfr{relEq} \rfr{T} \cusand y \rfr{relEq} \rfr{T} \end{cases}\mr) \cusnum{\cusend}{1}
\end{align}P
\end{tcolorbox}


\subsection{P: Boolean Algebra}
\begin{tcolorbox}[breakable, enhanced]
\begin{align}
    \dfn{booleanAlgebra} [\ml(\top, \bot, \otimes, \oplus, \ominus\mr)][] \cusnum{\rfr{defEq}}{1} \thmlink{}{POS-LCom} \ml(\ml(x \otimes y \rfr{relEq} y \otimes x\mr) \cusnum{\cusand}{1} \ml(x \oplus y \rfr{relEq} y \oplus x\mr)\mr) \eqComment{Commutative} \cusnum{\cusand}{1}
\m \thmlink{}{POS-LDis} \ml(\ml(x \otimes \ml(y \oplus z\mr) \rfr{relEq} \ml(x \otimes y\mr) \oplus \ml(x \otimes z\mr)\mr) \cusnum{\cusand}{1} \ml(x \oplus \ml(y \otimes z\mr) \rfr{relEq} \ml(x \oplus y\mr) \otimes \ml(x \oplus z\mr)\mr)\mr) \eqComment{Distributive} \cusnum{\cusand}{1}
\m \thmlink{}{POS-LIdn} \ml(\ml(x \otimes \top \rfr{relEq} x\mr) \cusnum{\cusand}{1} \ml(x \oplus \bot \rfr{relEq} x\mr)\mr) \eqComment{Identity} \cusnum{\cusand}{1}
\m \thmlink{}{POS-LCmp} \ml(\ml(x \otimes \ml(\ominus x\mr) \rfr{relEq} \bot\mr) \cusnum{\cusand}{1} \ml(x \oplus \ml(\ominus x\mr) \rfr{relEq} \top\mr)\mr) \eqComment{Complement} \cusnum{\cusend}{1}
\m \TODO{\eqComment{Note: I sometimes get too lazy to refer to $POS-LCom$.}} \img{my-b.jpg} 
    \n \thmlink{}{INS-LBAl} \ml(\rfr{booleanAlgebra} [\rfr{T}, \rfr{F}, \rfr{AND}, \rfr{OR}, \rfr{NOT}][]\mr)
\m \eqComment{Proven by way of cases or truth tables}
    \n \thmlink{}{THM-Dual-1} \Bigg(\rfr{booleanAlgebra} [\ml(\rfr{T}, \rfr{F}, \rfr{AND}, \rfr{OR}, \rfr{NOT}\mr)][] \cusnum{\rfr{IFF}}{1}
\m \ml(\ml(x \rfr{AND} y \rfr{relEq} y \rfr{AND} x\mr) \cusnum{\cusand}{1} \ml(x \rfr{OR} y \rfr{relEq} y \rfr{OR} x\mr)\mr) \eqComment{Commutative} \cusnum{\cusand}{1}
\m \ml(\ml(x \rfr{AND} \ml(y \rfr{OR} z\mr) \rfr{relEq} \ml(x \rfr{AND} y\mr) \rfr{OR} \ml(x \rfr{AND} z\mr)\mr) \cusnum{\cusand}{1} \ml(x \rfr{OR} \ml(y \rfr{AND} z\mr) \rfr{relEq} \ml(x \rfr{OR} y\mr) \rfr{AND} \ml(x \rfr{OR} z\mr)\mr)\mr) \eqComment{Distributive} \cusnum{\cusand}{1}
\m \ml(\ml(x \rfr{AND} \rfr{T} \rfr{relEq} x\mr) \cusnum{\cusand}{1} \ml(x \rfr{OR} \rfr{F} \rfr{relEq} x\mr)\mr) \eqComment{Identity} \cusnum{\cusand}{1}
\m \ml(\ml(x \rfr{AND} \rfr{NOT} x \rfr{relEq} \rfr{F}\mr) \cusnum{\cusand}{1} \ml(x \rfr{OR} \rfr{NOT} x \rfr{relEq} \rfr{T}\mr)\mr) \eqComment{Complement} \cusnum{\cusend}{1} \cusnum{\rfr{IFF}}{2}
\m \ml(\ml(x \rfr{OR} y \rfr{relEq} y \rfr{OR} x\mr) \cusnum{\cusand}{2} \ml(x \rfr{AND} y \rfr{relEq} y \rfr{AND} x\mr)\mr) \eqComment{Reordered Commutative} \cusnum{\cusand}{2}
\m \ml(\ml(x \rfr{OR} \ml(y \rfr{AND} z\mr) \rfr{relEq} \ml(x \rfr{OR} y\mr) \rfr{AND} \ml(x \rfr{OR} z\mr)\mr) \cusnum{\cusand}{2} \ml(x \rfr{AND} \ml(y \rfr{OR} z\mr) \rfr{relEq} \ml(x \rfr{AND} y\mr) \rfr{OR} \ml(x \rfr{AND} z\mr)\mr)\mr) \eqComment{Reordered Distributive} \cusnum{\cusand}{2}
\m \ml(\ml(x \rfr{OR} \rfr{F} \rfr{relEq} x\mr) \cusnum{\cusand}{2} \ml(x \rfr{AND} \rfr{T} \rfr{relEq} x\mr)\mr) \eqComment{Reordered Identity} \cusnum{\cusand}{2}
\m \ml(\ml(x \rfr{OR} \rfr{NOT} x \rfr{relEq} \rfr{T}\mr) \cusnum{\cusand}{2} \ml(x \rfr{AND} \rfr{NOT} x \rfr{relEq} \rfr{F}\mr)\mr) \eqComment{Reordered Complement} \cusnum{\cusend}{2} \rfr{IFF}
\m \rfr{booleanAlgebra} [\ml(\rfr{F}, \rfr{T}, \rfr{OR}, \rfr{AND}, \rfr{NOT}\mr)][]\Bigg)
\m \thmlink{THM-Dual-1}{THM-Dual-2} \ml(\rfr{booleanAlgebra} [\ml(\rfr{T}, \rfr{F}, \rfr{AND}, \rfr{OR}, \rfr{NOT}\mr)][] \rfr{IFF} \rfr{booleanAlgebra} [\ml(\rfr{F}, \rfr{T}, \rfr{OR}, \rfr{AND}, \rfr{NOT}\mr)][]\mr)
\m \thmlink{THM-Dual-2, INS-LBAl}{THM-Dual} \rfr{booleanAlgebra} [\ml(\rfr{F}, \rfr{T}, \rfr{OR}, \rfr{AND}, \rfr{NOT}\mr)][]\ml(\mr)
\m \eqComment{Boolean Algebra Duality follows from the swap symmetry of $\ml(\rfr{AND}, \rfr{T}\mr)$ and $\ml(\rfr{OR}, \rfr{F}\mr)$ within the axioms}
    \n \thmlink{}{THM-LUNt-1} \ml(\ml(x \rfr{OR} y \rfr{relEq} \rfr{T} \rfr{relEq} x \rfr{OR} z\mr) \rfr{AND} \ml(x \rfr{AND} y \rfr{relEq} \rfr{F} \rfr{relEq} x \rfr{AND} z\mr)\mr) \cusnum{\rfr{IF}}{1}
\m \thmlink{INS-LBAl, POS-LIdn}{THM-LUNt-2} \ml(y \rfr{relEq} y \rfr{AND} \rfr{T}\mr) \cusnum{\cusand}{1}
\m \thmlink{THM-LUNt-1}{THM-LUNt-3} \ml(y \rfr{AND} \rfr{T} \rfr{relEq} y \rfr{AND} \ml(x \rfr{OR} z\mr)\mr) \cusnum{\cusand}{1}
\m \thmlink{POS-LDis}{THM-LUNt-4} \ml(y \rfr{AND} \ml(x \rfr{OR} z\mr) \rfr{relEq} \ml(y \rfr{AND} x\mr) \rfr{OR} \ml(y \rfr{AND} z\mr)\mr) \cusnum{\cusand}{1}
\m \thmlink{POS-LCom, THM-LUNt-4}{THM-LUNt-5} \ml(\ml(y \rfr{AND} x\mr) \rfr{OR} \ml(y \rfr{AND} z\mr) \rfr{relEq} \ml(x \rfr{AND} z\mr) \rfr{OR} \ml(y \rfr{AND} z\mr)\mr) \cusnum{\cusand}{1}
\m \thmlink{POS-LCom, POS-LDis}{THM-LUNt-6} \ml(\ml(x \rfr{AND} z\mr) \rfr{OR} \ml(y \rfr{AND} z\mr) \rfr{relEq} z \rfr{AND} \ml(x \rfr{OR} y\mr)\mr) \cusnum{\cusand}{1}
\m \thmlink{THM-LUNt-1}{THM-LUNt-7} \ml(z \rfr{AND} \ml(x \rfr{OR} y\mr) \rfr{relEq} z \rfr{AND} \rfr{T}\mr) \cusnum{\cusand}{1}
\m \thmlink{POS-LIdn}{THM-LUNt-8} \ml(z \rfr{AND} \rfr{T} \rfr{relEq} z\mr) \cusnum{\cusend}{1}
\m \thmlink{THM-LUNt-1, THM-LUNt-2, THM-LUNt-3, THM-LUNt-4, THM-LUNt-5, THM-LUNt-6, THM-LUNt-7, THM-LUNt-8}{THM-LUNt} \ml(\ml(\ml(x \rfr{OR} y \rfr{relEq} \rfr{T} \rfr{relEq} x \rfr{OR} z\mr) \rfr{AND} \ml(x \rfr{AND} y \rfr{relEq} \rfr{F} \rfr{relEq} x \rfr{AND} z\mr)\mr) \rfr{IF} \ml(y \rfr{relEq} z\mr)\mr)
\m \eqComment{Uniqueness of Complements}
    \n \thmlink{INS-LBAl, POS-LIdn}{THM-LDom-1} \ml(x \rfr{OR} \rfr{T} \rfr{relEq} \ml(x \rfr{OR} \rfr{T}\mr) \rfr{AND} \rfr{T}\mr)
\m \thmlink{POS-LCmp}{THM-LDom-2} \ml(\ml(x \rfr{OR} \rfr{T}\mr) \rfr{AND} \rfr{T} \rfr{relEq} \ml(x \rfr{OR} \rfr{T}\mr) \rfr{AND} \ml(x \rfr{OR} \rfr{NOT} x\mr)\mr)
\m \thmlink{POS-LDis}{THM-LDom-3} \ml(\ml(x \rfr{OR} \rfr{T}\mr) \rfr{AND} \ml(x \rfr{OR} \rfr{NOT} x\mr) \rfr{relEq} x \rfr{OR} \ml(\rfr{T} \rfr{AND} \rfr{NOT} x\mr)\mr)
\m \thmlink{POS-LIdn}{THM-LDom-4} \ml(x \rfr{OR} \ml(\rfr{T} \rfr{AND} \rfr{NOT} x\mr) \rfr{relEq} x \rfr{OR} \rfr{NOT} x\mr)
\m \thmlink{POS-LCmp}{THM-LDom-5} \ml(x \rfr{OR} \rfr{NOT} x \rfr{relEq} \rfr{T}\mr)
\m \thmlink{THM-LDom-1, THM-LDom-2, THM-LDom-3, THM-LDom-4, THM-LDom-5}{THM-LDom-6} \ml(x \rfr{OR} \rfr{T} \rfr{relEq} \rfr{T}\mr)
\m \thmlink{THM-LDom-6, THM-Dual}{THM-LDom} \ml(\ml(x \rfr{OR} \rfr{T} \rfr{relEq} \rfr{T}\mr) \cusand \ml(x \rfr{AND} \rfr{F} \rfr{relEq} \rfr{F}\mr)\mr)
\m \eqComment{Domination}
    \n \thmlink{INS-LBAl, POS-LIdn}{THM-LIdm-1} \ml(x \rfr{OR} x \rfr{relEq} \ml(x \rfr{OR} x\mr) \rfr{AND} \rfr{T}\mr) 
\m \thmlink{POS-LCmp}{THM-LIdm-2} \ml(\ml(x \rfr{OR} x\mr) \rfr{AND} \rfr{T} \rfr{relEq} \ml(x \rfr{OR} x\mr) \rfr{AND} \ml(x \rfr{OR} \rfr{NOT} x\mr)\mr) 
\m \thmlink{POS-LDis}{THM-LIdm-3} \ml(\ml(x \rfr{OR} x\mr) \rfr{AND} \ml(x \rfr{OR} \rfr{NOT} x\mr) \rfr{relEq} x \rfr{AND} \ml(x \rfr{OR} \rfr{NOT} x\mr)\mr) 
\m \thmlink{POS-LCmp}{THM-LIdm-4} \ml(x \rfr{AND} \ml(x \rfr{OR} \rfr{NOT} x\mr) \rfr{relEq} x \rfr{AND} \rfr{T}\mr) 
\m \thmlink{POS-LIdn}{THM-LIdm-5} \ml(x \rfr{AND} \rfr{T} \rfr{relEq} x\mr) 
\m \thmlink{THM-LIdm-1, THM-LIdm-2, THM-LIdm-3, THM-LIdm-4, THM-LIdm-5}{THM-LIdm-6} \ml(x \rfr{OR} x \rfr{relEq} x\mr) 
\m \thmlink{THM-LIdm-6, THM-Dual}{THM-LIdm} \ml(\ml(x \rfr{OR} x \rfr{relEq} x\mr) , \ml(x \rfr{AND} x \rfr{relEq} x\mr)\mr) 
\m \eqComment{Idempotent}
    \n \thmlink{INS-LBAl, POS-LIdn}{THM-LInv-1} \ml(\rfr{NOT} \rfr{NOT} x \rfr{relEq} \rfr{NOT} \rfr{NOT} x \rfr{OR} \rfr{F}\mr)
\m \thmlink{POS-LCmp}{THM-LInv-2} \ml(\rfr{NOT} \rfr{NOT} x \rfr{OR} \rfr{F} \rfr{relEq} \rfr{NOT} \rfr{NOT} x \rfr{OR} \ml(x \rfr{AND} \rfr{NOT} x\mr)\mr) 
\m \thmlink{POS-LDis}{THM-LInv-3} \ml(\rfr{NOT} \rfr{NOT} x \rfr{OR} \ml(x \rfr{AND} \rfr{NOT} x\mr)\rfr{relEq} \ml(\rfr{NOT} \rfr{NOT} x \rfr{OR} x\mr) \rfr{AND} \ml(\rfr{NOT} \rfr{NOT} x \rfr{OR} \rfr{NOT} x\mr)\mr) 
\m \thmlink{POS-LCmp}{THM-LInv-4} \ml(\ml(\rfr{NOT} \rfr{NOT} x \rfr{OR} x\mr) \rfr{AND} \ml(\rfr{NOT} \rfr{NOT} x \rfr{OR} \rfr{NOT} x\mr) \rfr{relEq} \ml(\rfr{NOT} \rfr{NOT} x \rfr{OR} x\mr) \rfr{AND} \rfr{T}\mr) 
\m \thmlink{POS-LCmp}{THM-LInv-5} \ml(\ml(\rfr{NOT} \rfr{NOT} x \rfr{OR} x\mr) \rfr{AND} \rfr{T} \rfr{relEq} \ml(\rfr{NOT} \rfr{NOT} x \rfr{OR} x\mr) \rfr{AND} \ml(x \rfr{OR} \rfr{NOT} x\mr)\mr) 
\m \thmlink{POS-LDis}{THM-LInv-6} \ml(\ml(\rfr{NOT} \rfr{NOT} x \rfr{OR} x\mr) \rfr{AND} \ml(x \rfr{OR} \rfr{NOT} x\mr) \rfr{relEq} x \rfr{OR} \ml(\rfr{NOT} \rfr{NOT} x \rfr{AND} \rfr{NOT} x\mr)\mr) 
\m \thmlink{POS-LCmp}{THM-LInv-7} \ml(x \rfr{OR} \ml(\rfr{NOT} \rfr{NOT} x \rfr{AND} \rfr{NOT} x\mr)\rfr{relEq} x \rfr{OR} \rfr{F}\mr) 
\m \thmlink{POS-LIdn}{THM-LInv-8} \ml(x \rfr{OR} \rfr{F} \rfr{relEq} x\mr) 
\m \thmlink{THM-LInv-1, THM-LInv-2, THM-LInv-3, THM-LInv-4, THM-LInv-5, THM-LInv-6, THM-LInv-7, THM-LInv-8}{THM-LInv} \ml(\rfr{NOT} \rfr{NOT} x \rfr{relEq} x\mr) 
\m \eqComment{Involution}
    \n \thmlink{INS-LBAl, POS-LIdn}{THM-LAbs-1} \ml(x \rfr{OR} \ml(x \rfr{AND} y\mr) \rfr{relEq} \ml(x \rfr{AND} \rfr{T}\mr) \rfr{OR} \ml(x \rfr{AND} y\mr)\mr) 
\m \thmlink{POS-LDis}{THM-LAbs-2} \ml(\ml(x \rfr{AND} \rfr{T}\mr) \rfr{OR} \ml(x \rfr{AND} y\mr) \rfr{relEq} x \rfr{AND} \ml(\rfr{T} \rfr{OR} y\mr)\mr) 
\m \thmlink{THM-LDom}{THM-LAbs-3} \ml(x \rfr{AND} \ml(\rfr{T} \rfr{OR} y\mr) \rfr{relEq} x \rfr{AND} \rfr{T}\mr) 
\m \thmlink{POS-LIdn}{THM-LAbs-4} \ml(x \rfr{AND} \rfr{T} \rfr{relEq} x\mr) 
\m \thmlink{THM-LAbs-1, THM-LAbs-2, THM-LAbs-3, THM-LAbs-4}{THM-LAbs-5} \ml(x \rfr{OR} \ml(x \rfr{AND} y\mr) \rfr{relEq} x\mr) 
\m \thmlink{THM-LAbs-5, THM-Dual}{THM-LAbs} \ml(\ml(x \rfr{OR} \ml(x \rfr{AND} y\mr) \rfr{relEq} x\mr) \cusand \ml(x \rfr{AND} \ml(x \rfr{OR} y\mr) \rfr{relEq} x\mr)\mr) 
\m \eqComment{Absorption}
    \n \thmlink{}{THM-LAsc-1} \ml(\ml(A \rfr{relEq} x \rfr{OR} \ml(y \rfr{OR} z\mr)\mr) \cusand \ml(B \rfr{relEq} \ml(x \rfr{OR} y\mr) \rfr{OR} z\mr)\mr) \cusnum{\rfr{IF}}{1}
\m \thmlink{THM-LAsc-1}{THM-LAsc-2} \ml(x \rfr{AND} A \rfr{relEq} x \rfr{AND} \ml(x \rfr{OR} \ml(y \rfr{OR} z\mr)\mr)\mr) \cusnum{\cusand}{1} 
\m \thmlink{THM-LAbs}{THM-LAsc-3} \ml(x \rfr{AND} \ml(x \rfr{OR} \ml(y \rfr{OR} z\mr)\mr) \rfr{relEq} x\mr) \cusnum{\cusand}{1}
\m \thmlink{THM-LAsc-1}{THM-LAsc-4} \ml(x \rfr{AND} B \rfr{relEq} x \rfr{AND} \ml(\ml(x \rfr{OR} y\mr) \rfr{OR} z\mr)\mr) \cusnum{\cusand}{1}
\m \thmlink{INS-LBAl, POS-LDis}{THM-LAsc-5} \ml(x \rfr{AND} \ml(\ml(x \rfr{OR} y\mr) \rfr{OR} z\mr) \rfr{relEq} \ml(x \rfr{AND} \ml(x \rfr{OR} y\mr)\mr) \rfr{OR} \ml(x \rfr{AND} z\mr)\mr) \cusnum{\cusand}{1}
\m \thmlink{THM-LAbs}{THM-LAsc-6} \ml(\ml(x \rfr{AND} \ml(x \rfr{OR} y\mr)\mr) \rfr{OR} \ml(x \rfr{AND} z\mr) \rfr{relEq} x \rfr{OR} \ml(x \rfr{AND} z\mr)\mr) \cusnum{\cusand}{1}
\m \thmlink{THM-LAbs}{THM-LAsc-7} \ml(x \rfr{OR} \ml(x \rfr{AND} z\mr) \rfr{relEq} x\mr) \cusnum{\cusand}{1}
\m \thmlink{THM-LAbs}{THM-LAsc-8} \ml(\ml(x \rfr{AND} \ml(x \rfr{OR} y\mr)\mr) \rfr{OR} \ml(x \rfr{AND} z\mr) \rfr{relEq} x \rfr{OR} \ml(x \rfr{AND} z\mr)\mr) \cusnum{\cusand}{1}
\m \thmlink{THM-LAsc-2, THM-LAsc-3, THM-LAsc-4, THM-LAsc-5, THM-LAsc-6, THM-LAsc-7, THM-LAsc-8}{THM-LAsc-9} \ml(x \rfr{AND} A \rfr{relEq} x \rfr{relEq} x \rfr{AND} B\mr) \cusnum{\cusand}{1}
\m \thmlink{THM-LAsc-1}{THM-LAsc-10} \ml(\rfr{NOT} x \rfr{AND} A \rfr{relEq} \rfr{NOT} x \rfr{AND} \ml(x \rfr{OR} \ml(y \rfr{OR} z\mr)\mr)\mr) \cusnum{\cusand}{1}
\m \thmlink{INS-LBAl, POS-LDis}{THM-LAsc-11} \ml(\rfr{NOT} x \rfr{AND} \ml(x \rfr{OR} \ml(y \rfr{OR} z\mr)\mr) \rfr{relEq} \ml(\rfr{NOT} x \rfr{AND} x\mr) \rfr{OR} \ml(\rfr{NOT} x \rfr{AND} \ml(y + z\mr)\mr)\mr) \cusnum{\cusand}{1}
\m \thmlink{INS-LBAl, POS-LCmp}{THM-LAsc-12} \ml(\ml(\rfr{NOT} x \rfr{AND} x\mr) \rfr{OR} \ml(\rfr{NOT} x \rfr{AND} \ml(y \rfr{OR} z\mr)\mr) \rfr{relEq} \rfr{F} \rfr{OR} \ml(\rfr{NOT} x \rfr{AND} \ml(y \rfr{OR} z\mr)\mr)\mr) \cusnum{\cusand}{1}
\m \thmlink{INS-LBAl, POS-LIdn}{THM-LAsc-13} \ml(\rfr{F} \rfr{OR} \ml(\rfr{NOT} x \rfr{AND} \ml(y + z\mr)\mr) \rfr{relEq} \rfr{NOT} x \rfr{AND} \ml(y \rfr{OR} z\mr)\mr) \cusnum{\cusand}{1}
\m \thmlink{THM-LAsc-1}{THM-LAsc-14} \ml(\rfr{NOT} x \rfr{AND} B \rfr{relEq} \rfr{NOT} x \rfr{AND} \ml(\ml(x \rfr{OR} y\mr) \rfr{OR} z\mr)\mr) \cusnum{\cusand}{1}
\m \thmlink{INS-LBAl, POS-LDis}{THM-LAsc-15} \ml(\rfr{NOT} x \rfr{AND} \ml(\ml(x \rfr{OR} y\mr) \rfr{OR} z\mr) \rfr{relEq} \ml(\rfr{NOT} x \rfr{AND} \ml(x \rfr{OR} y\mr)\mr) \rfr{OR} \ml(\rfr{NOT} x \rfr{AND} z\mr)\mr) \cusnum{\cusand}{1}
\m \thmlink{INS-LBAl, POS-LDis}{THM-LAsc-16} \ml(\ml(\rfr{NOT} x \rfr{AND} \ml(x \rfr{OR} y\mr)\mr) \rfr{OR} \ml(\rfr{NOT} x \rfr{AND} z\mr) \rfr{relEq} \ml(\ml(\rfr{NOT} x \rfr{AND} x\mr) \rfr{OR} \ml(\rfr{NOT} x \rfr{AND} y\mr)\mr) \rfr{OR} \ml(\rfr{NOT} x \rfr{AND} z\mr)\mr) \cusnum{\cusand}{1}
\m \thmlink{INS-LBAl, POS-LCmp}{THM-LAsc-17} \ml(\ml(\ml(\rfr{NOT} x \rfr{AND} x\mr) \rfr{OR} \ml(\rfr{NOT} x \rfr{AND} y\mr)\mr) \rfr{OR} \ml(\rfr{NOT} x \rfr{AND} z\mr) \rfr{relEq} \ml(\rfr{F} \rfr{OR} \ml(\rfr{NOT} x \rfr{AND} y\mr)\mr) \rfr{OR} \ml(\rfr{NOT} x \rfr{AND} z\mr)\mr) \cusnum{\cusand}{1}
\m \thmlink{INS-LBAl, POS-LIdn}{THM-LAsc-18} \ml(\ml(\rfr{F} \rfr{OR} \ml(\rfr{NOT} x \rfr{AND} y\mr)\mr) \rfr{OR} \ml(\rfr{NOT} x \rfr{AND} z\mr) \rfr{relEq} \ml(\rfr{NOT} x \rfr{AND} y\mr) \rfr{OR} \ml(\rfr{NOT} x \rfr{AND} z\mr)\mr) \cusnum{\cusand}{1}
\m \thmlink{INS-LBAl, POS-LDis}{THM-LAsc-19} \ml(\ml(\rfr{NOT} x \rfr{AND} y\mr) \rfr{OR} \ml(\rfr{NOT} x \rfr{AND} z\mr) \rfr{relEq} \rfr{NOT} x \rfr{AND} \ml(y \rfr{OR} z\mr)\mr) \cusnum{\cusand}{1}
\m \thmlink{THM-LAsc-10, THM-LAsc-11, THM-LAsc-12, THM-LAsc-13, THM-LAsc-14, THM-LAsc-15, THM-LAsc-16, THM-LAsc-17, THM-LAsc-18, THM-LAsc-19}{THM-LAsc-20} \ml(\rfr{NOT} x \rfr{AND} A \rfr{relEq} \rfr{NOT} x \rfr{AND} \ml(y \rfr{OR} z\mr) \rfr{relEq} \rfr{NOT} x \rfr{AND} B\mr) \cusnum{\cusand}{1}
\m \thmlink{INS-LBAl, POS-LDis}{THM-LAsc-21} \ml(A \rfr{relEq} A \rfr{AND} \rfr{T}\mr) \cusnum{\cusand}{1}
\m \thmlink{INS-LBAl, POS-LCmp}{THM-LAsc-22} \ml(A \rfr{AND} \rfr{T} \rfr{relEq} A \rfr{AND} \ml(x \rfr{OR} \rfr{NOT} x\mr)\mr) \cusnum{\cusand}{1}
\m \thmlink{INS-LBAl, POS-LDis}{THM-LAsc-23} \ml(A \rfr{AND} \ml(x \rfr{OR} \rfr{NOT} x\mr) \rfr{relEq} \ml(x \rfr{AND} A\mr) \rfr{OR} \ml(\rfr{NOT} x \rfr{AND} A\mr)\mr) \cusnum{\cusand}{1}
\m \thmlink{THM-LAsc-9}{THM-LAsc-24} \ml(\ml(x \rfr{AND} A\mr) \rfr{OR} \ml(\rfr{NOT} x \rfr{AND} A\mr) \rfr{relEq} \ml(x \rfr{AND} B\mr) \rfr{OR} \ml(\rfr{NOT} x \rfr{AND} A\mr)\mr) \cusnum{\cusand}{1}
\m \thmlink{THM-LAsc-20}{THM-LAsc-25} \ml(\ml(x \rfr{AND} B\mr) \rfr{OR} \ml(\rfr{NOT} x \rfr{AND} A\mr) \rfr{relEq} \ml(x \rfr{AND} B\mr) \rfr{OR} \ml(\rfr{NOT} x \rfr{AND} B\mr)\mr) \cusnum{\cusand}{1}
\m \thmlink{INS-LBAl, POS-LDis}{THM-LAsc-26} \ml(\ml(x \rfr{AND} B\mr) \rfr{OR} \ml(\rfr{NOT} x \rfr{AND} B\mr) \rfr{relEq} B \rfr{AND} \ml(x \rfr{OR} \rfr{NOT} x\mr)\mr) \cusnum{\cusand}{1}
\m \thmlink{INS-LBAl, POS-LCmp}{THM-LAsc-27} \ml(B \rfr{AND} \ml(x \rfr{OR} \rfr{NOT} x\mr) \rfr{relEq} B \rfr{AND} \rfr{T}\mr) \cusnum{\cusand}{1}
\m \thmlink{INS-LBAl, POS-LIdn}{THM-LAsc-27} \ml(B \rfr{AND} \rfr{T} \rfr{relEq} B\mr) \cusnum{\cusand}{1}
\m \thmlink{THM-LAsc-21, THM-LAsc-22, THM-LAsc-23, THM-LAsc-24, THM-LAsc-25, THM-LAsc-26, THM-LAsc-27}{THM-LAsc-28} \ml(A \rfr{relEq} B\mr) \cusnum{\cusand}{1}
\m \thmlink{THM-LAsc-28, THM-LAsc-1}{THM-LAsc-29} \ml(x \rfr{OR} \ml(y \rfr{OR} z\mr) \rfr{relEq} \ml(x \rfr{OR} y\mr) \rfr{OR} z\mr) \cusnum{\cusend}{1}
\m \thmlink{THM-LAsc-29, THM-Dual}{THM-LAsc} \ml(\ml(x \rfr{OR} \ml(y \rfr{OR} z\mr) \rfr{relEq} \ml(x \rfr{OR} y\mr) \rfr{OR} z\mr) \cusand \ml(x \rfr{AND} \ml(y \rfr{AND} z\mr) \rfr{relEq} \ml(x \rfr{AND} y\mr) \rfr{AND} z\mr)\mr) 
\m \eqComment{Associative}
    \n \thmlink{INS-LBAl, POS-LDis}{THM-LDMr-1} \ml(\ml(x \rfr{OR} y\mr) \rfr{OR} \ml(\rfr{NOT} x \rfr{AND} \rfr{NOT} y\mr) \rfr{relEq} \ml(\ml(x \rfr{OR} y\mr) \rfr{OR} \rfr{NOT} x\mr) \rfr{AND} \ml(\ml(x \rfr{OR} y\mr) \rfr{OR} \rfr{NOT} y\mr)\mr) 
\m \thmlink{POS-LCom, THM-LAsc}{THM-LDMr-2} \ml(\ml(\ml(x \rfr{OR} y\mr) \rfr{OR} \rfr{NOT} x\mr) \rfr{AND} \ml(\ml(x \rfr{OR} y\mr) \rfr{OR} \rfr{NOT} y\mr) \rfr{relEq} \ml(\ml(x \rfr{OR} \rfr{NOT} x\mr) \rfr{OR} y\mr) \rfr{AND} \ml(\ml(\rfr{NOT} y \rfr{OR} y\mr) \rfr{OR} x\mr)\mr) 
\m \thmlink{INS-LBAl, POS-LCmp}{THM-LDMr-3} \ml(\ml(\ml(x \rfr{OR} \rfr{NOT} x\mr) \rfr{OR} y\mr) \rfr{AND} \ml(\ml(\rfr{NOT} y \rfr{OR} y\mr) \rfr{OR} x\mr) \rfr{relEq} \ml(\rfr{T} \rfr{OR} y\mr) \rfr{AND} \ml(\rfr{T} \rfr{OR} x\mr)\mr) 
\m \thmlink{THM-LDom}{THM-LDMr-4} \ml(\ml(\rfr{T} \rfr{OR} y\mr) \rfr{AND} \ml(\rfr{T} \rfr{OR} x\mr) \rfr{relEq} \rfr{T} \rfr{AND} \rfr{T}\mr) 
\m \thmlink{THM-LIdm}{THM-LDMr-5} \ml(\rfr{T} \rfr{AND} \rfr{T} \rfr{relEq} \rfr{T}\mr) 
\m \thmlink{THM-LDMr-1, THM-LDMr-2, THM-LDMr-3, THM-LDMr-4, THM-LDMr-5}{THM-LDMr-6} \ml(\ml(x \rfr{OR} y\mr) \rfr{OR} \ml(\rfr{NOT} x \rfr{AND} \rfr{NOT} y\mr) \rfr{relEq} \rfr{T}\mr) 
\m \thmlink{INS-LBAl, POS-LDis}{THM-LDMr-7} \ml(\ml(x \rfr{OR} y\mr) \rfr{AND} \ml(\rfr{NOT} x \rfr{AND} \rfr{NOT} y\mr) \rfr{relEq} \ml(x \rfr{AND} \rfr{NOT} x \rfr{AND} \rfr{NOT} y\mr) \rfr{OR} \ml(y \rfr{AND} \rfr{NOT} x \rfr{AND} \rfr{NOT} y\mr)\mr) 
\m \thmlink{POS-LCom, THM-LAsc}{THM-LDMr-8} \ml(\ml(x \rfr{AND} \rfr{NOT} x \rfr{AND} \rfr{NOT} y\mr) \rfr{OR} \ml(y \rfr{AND} \rfr{NOT} x \rfr{AND} \rfr{NOT} y\mr) \rfr{relEq} \ml(\ml(x \rfr{AND} \rfr{NOT} x\mr) \rfr{AND} \rfr{NOT} y\mr) \rfr{OR} \ml(\ml(y \rfr{AND} \rfr{NOT} y\mr) \rfr{AND} \rfr{NOT} x\mr)\mr) 
\m \thmlink{INS-LBAl, POS-LCmp}{THM-LDMr-9} \ml(\ml(\ml(x \rfr{AND} \rfr{NOT} x\mr) \rfr{AND} \rfr{NOT} y\mr) \rfr{OR} \ml(\ml(y \rfr{AND} \rfr{NOT} y\mr) \rfr{AND} \rfr{NOT} x\mr) \rfr{relEq} \ml(\rfr{F} \rfr{AND} \rfr{NOT} y\mr) \rfr{OR} \ml(\rfr{F} \rfr{AND} \rfr{NOT} x\mr)\mr) 
\m \thmlink{THM-LDom}{THM-LDMr-10} \ml(\ml(\rfr{F} \rfr{AND} \rfr{NOT} y\mr) \rfr{OR} \ml(\rfr{F} \rfr{AND} \rfr{NOT} x\mr) \rfr{relEq} \rfr{F} \rfr{OR} \rfr{F}\mr) 
\m \thmlink{THM-LIdm}{THM-LDMr-11} \ml(\rfr{F} \rfr{OR} \rfr{F} \rfr{relEq} \rfr{F}\mr) 
\m \thmlink{THM-LDMr-7, THM-LDMr-8, THM-LDMr-9, THM-LDMr-10, THM-LDMr-11}{THM-LDMr-12} \ml(\ml(x \rfr{OR} y\mr) \rfr{AND} \ml(\rfr{NOT} x \rfr{AND} \rfr{NOT} y\mr) \rfr{relEq} \rfr{F}\mr) 
\m \thmlink{THM-LDMr-6, THM-LDMr-12, POS-LCmp}{THM-LDMr-13} \ml(\ml(\ml(x \rfr{OR} y\mr) \rfr{OR} \ml(\rfr{NOT} x \rfr{AND} \rfr{NOT} y\mr) \rfr{relEq} \rfr{T} \rfr{relEq} \ml(x \rfr{OR} y\mr) \rfr{OR} \rfr{NOT} \ml(x \rfr{OR} y\mr)\mr) \cusand \ml(\ml(x \rfr{OR} y\mr) \rfr{AND} \ml(\rfr{NOT} x \rfr{AND} \rfr{NOT} y\mr) \rfr{relEq} \rfr{F} \rfr{relEq} \ml(x \rfr{OR} y\mr) \rfr{AND} \rfr{NOT} \ml(x \rfr{OR} y\mr)\mr)\mr) 
\m \thmlink{THM-LDMr-13, THM-LUNt}{THM-LDMr-14} \ml(\rfr{NOT} x \rfr{AND} \rfr{NOT} y \rfr{relEq} \rfr{NOT} \ml(x \rfr{OR} y\mr)\mr) 
\m \thmlink{THM-LDMr-14, THM-Dual}{THM-LDMr} \ml(\ml(\rfr{NOT} x \rfr{AND} \rfr{NOT} y \rfr{relEq} \rfr{NOT} \ml(x \rfr{OR} y\mr)\mr) \cusand \ml(\rfr{NOT} x \rfr{OR} \rfr{NOT} y \rfr{relEq} \rfr{NOT} \ml(x \rfr{AND} y\mr)\mr)\mr) 
\m \eqComment{Boolean De Morgan's Laws}
    \n \thmlink{operatorIF}{THM-CtrP-1} \ml(x \rfr{IF} y \rfr{relEq} \ml(\rfr{NOT} x\mr) \rfr{OR} y\mr)  
\m \thmlink{POS-LCom, THM-LInv}{THM-CtrP-2} \ml(\ml(\rfr{NOT} x\mr) \rfr{OR} y \rfr{relEq} \ml(\ml(\rfr{NOT} \rfr{NOT} y\mr) \rfr{OR} \ml(\rfr{NOT} x\mr)\mr)\mr)  
\m \thmlink{operatorIF}{THM-CtrP-3} \ml(\ml(\rfr{NOT} \rfr{NOT} y\mr) \rfr{OR} \ml(\rfr{NOT} x\mr) \rfr{relEq} \ml(\rfr{NOT} y\mr) \rfr{IF} \ml(\rfr{NOT} x\mr)\mr)  
\m \thmlink{THM-CtrP-1, THM-CtrP-2, THM-CtrP-3}{THM-CtrP} \ml(x \rfr{IF} y \rfr{relEq} \ml(\rfr{NOT} y\mr) \rfr{IF} \ml(\rfr{NOT} x\mr)\mr) 
\m \eqComment{Contrapositive Law} 
    \n \textbf{MISC IMPLICATION LAWS: } % http://www.cs.um.edu.mt/gordon.pace/Teaching/DiscreteMaths/Laws.pdf
\m \TODO{\ml(\rfr{T} \rfr{IF} x \rfr{relEq} x\mr)}
\m \TODO{\ml(\rfr{F} \rfr{IF} x \rfr{relEq} \rfr{T}\mr)}
\m \TODO{\ml(x \rfr{IF} \rfr{T} \rfr{relEq} \rfr{T}\mr)}
\m \TODO{\ml(x \rfr{IF} \rfr{F} \rfr{relEq} \rfr{NOT} x\mr)}  
\m \TODO{\ml(\ml(x \rfr{OR} y\mr) \rfr{IF} z\mr) \rfr{relEq} \ml(x \rfr{IF} z\mr) \rfr{AND} \ml(y \rfr{IF} z\mr)}
\m \TODO{\ml(x \rfr{IF} \ml(y \rfr{AND} z\mr) \rfr{relEq} \ml(x \rfr{IF} y\mr) \rfr{AND} \ml(x \rfr{IF} z\mr)\mr)} 
\m \TODO{\ml(x \rfr{IFF} y \rfr{relEq} x \rfr{IF} y \rfr{AND} \rfr{NOT} x \rfr{IF} \rfr{NOT} y\mr)}
\end{align}
\end{tcolorbox}


\subsection{Predicates, Sets, Tuples}
\TODO{undefined terms:} $%
   \cuspop{\_}{\_}, %
   \dfn{set}, %
   \dfn{relationIn}[\in], %
   \uset{\_}, %
$
\begin{tcolorbox}[breakable, enhanced]
\begin{align}
    \dfn{predicate} [P][] \rfr{defEq} \rfr{truth} [P(\free{v})][] 
    \n \dfn{universalQuantifier} [\dfn{FORALL}[\forall]][P] \cusnum{\rfr{defEq}}{1} (\rfr{predicate} [P][]) \cusnum{\cusand}{1} 
\m (\rfr{FORALL}_{\free{x}} (P(\free{x})) \rfr{relEq} P(\free{y})) \cusnum{\cusend}{1}
    \n \dfn{existentialQuantifier} [\dfn{EXISTS}[\exists]][Q, P] \rfr{defEq} (\rfr{EXISTS}_{\cuspop{x}{Q(x)}} (P(x)) \rfr{relEq} \rfr{NOT} \rfr{FORALL}_{\cuspop{x}{Q(x)}} (\rfr{NOT} P(x)))
    \n \dfn{uniquenessQuantifier} [\dfn{UNIQUE}[\exists!]][Q, P] \rfr{defEq} (\rfr{UNIQUE}_{\cuspop{x}{Q(x)}} (P(x)) \rfr{relEq} \rfr{EXISTS}_{\cuspop{x}{Q(x)}} (P(x) \rfr{AND} \rfr{NOT} \rfr{EXISTS}_{\cuspop{y}{Q(y)}} (P(y) \rfr{AND} \rfr{NOT} (y \rfr{relEq} x))))
    \n \dfn{relationSetEq} [\dfn{seteq}[\rfr{relEq}]][X, Y] \rfr{defEq} (\rfr{FORALL}_{\cuspop{z}{z \rfr{in} X \rfr{OR} z \rfr{in} Y}} (z \rfr{in} X \rfr{AND} z \rfr{in} Y))
    \n \dfn{operatorIntersection} [\dfn{intersect}[\bigcap]][X] \rfr{defEq} (z \rfr{in} \rfr{intersect}(X) \rfr{IFF} \rfr{FORALL}_{x \rfr{relationIn} X} (z \rfr{relationIn} x))
    \n \dfn{operatorUnion} [\dfn{union}[\bigcup]][X] \rfr{defEq} (z \rfr{in} \rfr{union}(X) \rfr{IFF} \rfr{EXISTS}_{x \rfr{relationIn} X} (z \rfr{relationIn} x))
    \n \dfn{orderedPair} [<x, y>][] == <x, y> = <a, b> iff x = a and y = b == \{\{x\},\{x,y\}\} %WATDO1st
\end{align}
\end{tcolorbox}
% TODO CODING STANDARDS - NAMING SCHEME
% TODO define all elementary structures, then return back to kinder level first order logic!
% TODO define notion of sets of sets of sets of ... mathcal standard

\begin{comment}

\TODO{use posets to define an abstract operator, and then use set structures to define universal operators}
\TODO{predicate shenanigans}
% http://planetmath.org/formalsystem
% http://planetmath.org/axiomsystemforpropositionallogic
% DEFINITION VS IMPLEMENTTION ??? WATDO
% TODO dont forget universal proposition operator theorem
\subsection{Axiom System for Propositional Logic}
\TODO{\textbf{undefined terms}:} $ 
  \dfn{set}, % 
  \dfn{infiniteSet}, %
  \dfn{finiteSet}, % 
  \dfn{countableSet}, %
  \dfn{orderedPair}, % https://math.stackexchange.com/a/62937
  \dfn{finiteSequence}, % http://planetmath.org/orderedtuplet
  \dfn{sequence}, % natural numbers ??
  \dfn{index}, % return an object from a sequence
  \dfn{length}, % return the length of a sequence
$ 
\begin{tcolorbox}[breakable, enhanced]
\begin{align} 
    \n AKA PROPOSITIONAL CALCULUS + SYNTAX
    \n \dfn{expression} [E][A] \rfr{defEq} (\rfr{set} [A][]) \cusand (\rfr{finiteSequence} [E][]) \cusand (for all i in len E index(E, i) in A)
\end{align}
\end{tcolorbox}


\begin{comment}

\section{Dry Run}
\subsection{NaiveMaster}
\TODO{\textbf{undefined terms}:} \rfr{defEq}, $%
   \dfn{set}, %
   \dfn{tuple}, %
   \dfn{element}, % 
   \dfn{nnumber}, %
   \dfn{-in}[\in], %
   \dfn{-subset}[\subseteq], %
   \dfn{-eq}[=], %
   \dfn{-not}[\not] \ , %
   \dfn{-psubset}[\subset], %
   \dfn{-union}[\cup], %
   \dfn{-intersection}[\cap], %
   \dfn{empty-set}[\emptyset], %
   \dfn{usetl}[\{], %
   \dfn{usetr}[\}], %
   \dfn{otupl}[\langle], %
   \dfn{otupr}[\rangle], %
   \dfn{st}[\mid], %
   \dfn{ktupset}[\widehat{\ }], %
   \dfn{cartesian}[\times], %
   \dfn{relation}, %
   \dfn{property}, %
   \dfn{binaryRelation}, %
     \\ %
   \dfn{domain}, %
   \dfn{range}, %
   \dfn{field}, %
   \dfn{universal}[\forall]. %
   \dfn{existential}[\exists], %
   \dfn{-and}[\land], %
   \dfn{-or}[\lor], %
   \dfn{-if}[\implies], %
   \dfn{-oif}[\impliedby], %
   \dfn{-iff}[\iff], % 
$
\begin{tcolorbox}[breakable, enhanced]
\begin{align}
\n \rfr{element} [x][] \rfr{-in} \rfr{set} [y][] %
%%%\rfr{-in}[\rfr{element} [x][]][\rfr{set} [y][]]
\m \eqComment{x belongs to y}
\n \rfr{set} [x][] \rfr{-subset} \rfr{set} [y][]
\m \eqComment{x is included in y}
\n (\rfr{set} [x][] \rfr{-eq} \rfr{set} [y][]) \rfr{defEq} (\rfr{set} [x][] \rfr{-subset} \rfr{set} [y][] \rfr{-and} \rfr{set} [y][] \rfr{-subset} \rfr{set} [x][])
\m \eqComment{x is the same set as y}
\n (\rfr{set} [x][] \rfr{-psubset} \rfr{set} [y][]) \rfr{defEq} (\rfr{set} [x][] \subseteq \rfr{set} [y][] \rfr{-and} \rfr{set} [x][] \rfr{-not} \rfr{-eq} \rfr{set} [y][])
\m \eqComment{x is a proper subset of y}
\n \rfr{set} [x][] \rfr{-union} \rfr{set} [y][]
\m \eqComment{all elements in x or y}
\n \rfr{set} [x][] \rfr{-intersection} \rfr{set} [y][]
\m \eqComment{all elements in x and y}
\n \dfn{disjoint} [x, y][] \rfr{defEq} \rfr{set} [x][] \rfr{-intersection} \rfr{set} [y][] \rfr{-eq} \rfr{empty-set}
\m \eqComment{disjoint sets do not intersect}
\n \rfr{set} [E][] \rfr{-eq} \rfr{usetl} \enlist{e}{n} \rfr{usetr}
\m \eqComment{unordered set containing $\enlist{e}{n}$}
\m \rfr{usetl} e_1, e_2, e_3 \rfr{usetr} \rfr{-eq} \rfr{usetl} e_3, e_1, e_2 \rfr{usetr}
\n \rfr{tuple} [E][] \rfr{-eq} \rfr{otupl} \enlist{e}{n} \rfr{otupr} 
\m \eqComment{ordered tuple containing $\enlist{e}{n}$}
\m \rfr{otupl} e_1, e_2, e_3 \rfr{otupr} \rfr{-not} \rfr{-eq} \rfr{otupl} e_2, e_3, e_1 \rfr{otupr}
\n \rfr{set} [X][] \rfr{ktupset} \rfr{nnumber} [k][]
\m \eqComment{set of all ordered k-tuples from the elements in X}
\m X \rfr{ktupset} 1 \rfr{-eq} \rfr{usetl} \enlist{e}{n} \rfr{usetr} \rfr{ktupset} 1 \rfr{-eq} \rfr{usetl} \rfr{otupl} e_1 \rfr{otupr}, \rfr{otupl} e_2 \rfr{otupr}, \rfr{otupl} e_3 \rfr{otupr}, \cdots, \rfr{otupl} e_n \rfr{otupr} \rfr{usetr} = \rfr{usetl} \enlist{e}{n} \rfr{usetr} \rfr{-eq} X
\n \rfr{set} [Y][] \rfr{times} \rfr{set} [Z][]
\m \eqComment{Cartesian product}
\n \rfr{relation} [R][S, k] \rfr{defEq} R \rfr{-subset} \rfr{set} [S][] \rfr{ktupset} \rfr{nnumber} [k][]
\m \eqComment{k-tuple relation R on the set S takes only tuples that satisfy some relation}
\n \rfr{property} [P][S] \rfr{defEq} \rfr{relation} [P][S, 1] \rfr{-subset} \rfr{set} [S][] \rfr{ktupset} 1 \rfr{-eq} S
\m \eqComment{property P of the set S}
\n \TODO{\textbf{use defeq to keep the object typing instead of }}
\m \TODO{\textbf{ref-eq which reduces it to a propositional truth value. }}
\m \TODO{\textbf{But, how do I chain ref-eq tho?}}
\n \rfr{binaryRelation} [B][S] \rfr{-eq} \rfr{relation} [B][S, 2] \rfr{-subset} \rfr{set} [S][] \rfr{ktupset} 2
\m x B y \rfr{-eq} \rfr{otupl} x, y \rfr{otupr} \rfr{-in} B 
\n \rfr{domain} [X][B, S] \rfr{-eq} \rfr{usetl} x \rfr{st} \rfr{otupl} x, y \rfr{otupr} \rfr{-in} \rfr{binaryRelation} [B][S] \rfr{usetr}
\n \rfr{range} [Y][B, S] \rfr{-eq} \rfr{usetl} y \rfr{st} \rfr{otupl} x, y \rfr{otupr} \rfr{-in} \rfr{binaryRelation} [B][S] \rfr{usetr}
\n \rfr{field} [F][B, S] \rfr{-eq} \rfr{domain} [X][B, S] \rfr{-union} \rfr{range} [Y][B, S]
\n \dfn{inverseRelation} [B^{-1}][B, S] \rfr{defEq} \rfr{usetl} \rfr{otupl} y, x \rfr{otupr} \rfr{st} \rfr{otupl} x, y \rfr{otupr} \rfr{-in} \rfr{binaryRelation} [B][S] \rfr{usetr}
\n \dfn{reflexive} [B][S] \rfr{defEq} \rfr{universal}_{x \rfr{-in} \rfr{field} [F][B, S]} (x B x)
\n \dfn{symmetric} [B][S] \rfr{defEq} \rfr{universal}_{x, y \in S} (x B y \rfr{-if} y B x)
\n \dfn{transitive} [B][S] \rfr{defEq} \rfr{universal}_{x, y, z \in S} ((x B y \rfr{-and} y B z) \rfr{-if} x B z)
\n \dfn{equivalenceRelation} [B][S] \rfr{defEq} (\rfr{reflexive} [B][S] \rfr{-and} \rfr{symmetric} [B][S] \rfr{-and} \rfr{transitive} [B][S])
\n \dfn{equivalenceClass} [[y]][y, B, S] \rfr{defEq} \{z \rfr{-in} \rfr{field} [F][B, S] \rfr{st} y B z\}
\m (\rfr{equivalenceClass} [[u]][u, B, S] \rfr{-eq} \rfr{equivalenceClass} [[v]][v, B, S]) \rfr{-iff} (u B v)
\m (\rfr{equivalenceClass} [[u]][u, B, S] \rfr{-not} \rfr{-eq} [[v]][v, B, S]) \rfr{-if} ([u] \rfr{-intersection} [v] \rfr{-eq} \rfr{emptyset})
\m \eqComment{a set can be partitioned by equivalence classes}
\n \dfn{function} [f][S] \rfr{defEq} (\rfr{universal}_{x, y, z \rfr{-in} S} (\rfr{otupl} x, y \rfr{otupr} \rfr{in} f \rfr{-and} \rfr{otupl} x, y \rfr{otupr} \rfr{-in} f) \rfr{-if} (y = z))
\end{align}
\end{tcolorbox}


\end{comment}

\end{document}
