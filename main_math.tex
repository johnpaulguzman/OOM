% CONTINUE 1090
% CONTINUE: PT - LINE 1000 PAGE 33
% CONTINUE: AA - LINE 660 PAGE 20
% NOTE DUPLICATE 439
% Operators dump: \odot$ \quad $\oplus$ \quad $\otimes$ \quad $\ominus$ \quad $\oslash
\documentclass[a4paper]{article}

%% Language and font encodings
\usepackage[english]{babel}
\usepackage[utf8x]{inputenc}
\usepackage[T1]{fontenc}

%% Sets page size and margins
\usepackage[a4paper,top=1.25cm,bottom=1.25cm,left=1.0cm,right=1.0cm,marginparwidth=1.25cm]{geometry}

%% Useful packages
\usepackage{amsmath}
\usepackage{graphicx}
\usepackage[colorinlistoftodos]{todonotes}
\usepackage[colorlinks=true, allcolors=blue]{hyperref}
\DeclareMathSizes{12}{30}{16}{12}
\usepackage[final]{pdfpages}
\usepackage{gensymb}
\usepackage{empheq}
\usepackage[section]{placeins}
\usepackage[most]{tcolorbox}
\usepackage{esint}
\usepackage{amssymb}
\usepackage{centernot}
\usepackage{multicol}
\usepackage{etoolbox}
\delimitershortfall-1sp
\usepackage{mleftright}
\mleftright
\def\ml{\mleft}
\def\mr{\mright}

\makeatletter
\newcommand{\defeq}{:=}
\newcommand{\cusand}{,}
\newcommand{\eqComment}[1]{\text{  \# #1}}
\newcommand{\eqSep}{\text{ ;  }}
\newcommand{\n}{\\[1.5ex] \hline \nonumber \\[0ex]}
\newcommand{\m}{\nonumber \\}
\newcommand{\field}[1]{\textbf{\textit{#1}}}
\newcommand*\features{}
\newcommand{\labeltarget}[1]{\Hy@raisedlink{\hypertarget{#1}{}}}
\newcommand{\dfn}[1]{\textcolor{teal}{#1}\labeltarget{#1}}
% ## BUGS MULTI dfn PER LINE ## \newcommand{\dfn}[1]{\textcolor{teal}{#1}\labeltarget{#1}\feature{#1}}
\newcommand{\rfr}[1]{\hyperlink{#1}{#1}}
\newcommand*\feature[1]
  {\ifx\features\empty
     \def\features{   \noindent \rfr{#1} \\}%
   \else
     \g@addto@macro\features{\rfr{#1} \\}%
   \fi}
\newcommand{\thm}[1]{\text{(THM) #1: }}
\newcommand\rfrlist[1]{%
    \forcsvlist{\rfrlist@item}{#1}
}
\newcommand\rfrlist@item[1]{\rfr{#1}\\}
\newcommand{\thmlink}[2]{{}_{\substack{\rfrlist{#1}}}^{\dfn{#2}} }
\makeatother

\title{Next-Next-Gen Notes \\
\large Object-Oriented Maths}
\author{JP Guzman}

\begin{document}
\maketitle
\allowdisplaybreaks

Format: $characteristic((subjects), (dependencies)) \iff (conditions(dependencies)) \land (conditions(subjects))$

Note: All weaker objects automatically induces notions inherited from stronger objects.

TODO assign free variables as parameters

TODO define || abs  cross-product and other missing refs

TODO distinguish new condition vs implied proposition
- separate propositions into new line thms

TODO silent link expressions!
- e.g. $backslash silentPL{PL_X}$

\thinmuskip=2mu % commas
\medmuskip=2mu % operators
\thickmuskip=2mu % equalities
\setlength{\belowdisplayskip}{0pt} \setlength{\belowdisplayshortskip}{0pt}
\setlength{\abovedisplayskip}{0pt} \setlength{\abovedisplayshortskip}{0pt}

% https://en.wikipedia.org/wiki/List_of_mathematical_abbreviations
% https://en.wikipedia.org/wiki/Propositional_calculus

\section{Logic and Set Theory}
\subsection{Truth shennanigans}
\begin{tcolorbox}[breakable, enhanced]
\begin{align}
   \dfn{truth}[t][] \defeq t = \begin{cases} \dfn{TRUE} \\ \dfn{FALSE}\end{cases} 
\n \dfn{statement}[s][] \defeq \rfr{correctSyntaxSemantics}[s][] 
\n \dfn{proposition}[s, t][] \defeq (\rfr{statement}[s][]) \cusand (\rfr{truth}[t][])
\n \dfn{operatorNOT}[\dfn{\lnot}][x] \defeq (\rfr{truth}[x][]) \cusand (\rfr{\lnot} x = \begin{cases} \rfr{TRUE} & x = \rfr{FALSE} \\ \rfr{FALSE} & x = \rfr{TRUE} \end{cases}) 
\n \dfn{operatorAND}[\dfn{\land}][x, y] \defeq (\rfr{truth}[x][]) \cusand (\rfr{truth}[y][]) \cusand (x \rfr{\land} y = \begin{cases} \rfr{FALSE} & x = \rfr{FALSE} \cusand y = \rfr{FALSE} \\ \rfr{FALSE} & x = \rfr{FALSE} \cusand y = \rfr{TRUE} \\ \rfr{FALSE} & x = \rfr{TRUE} \cusand y = \rfr{FALSE} \\ \rfr{TRUE} & x = \rfr{TRUE} \cusand y = \rfr{TRUE} \end{cases})
\n \dfn{operatorOR}[\dfn{\lor}][x, y] \defeq (\rfr{truth}[x][]) \cusand (\rfr{truth}[y][]) \cusand (x \rfr{\lor} y = \begin{cases} \rfr{FALSE} & x = \rfr{FALSE} \cusand y = \rfr{FALSE} \\ \rfr{TRUE} & x = \rfr{FALSE} \cusand y = \rfr{TRUE} \\ \rfr{TRUE} & x = \rfr{TRUE} \cusand y = \rfr{FALSE} \\ \rfr{TRUE} & x = \rfr{TRUE} \cusand y = \rfr{TRUE} \end{cases})
\n \dfn{operatorXOR}[\dfn{\veebar}][x, y] \defeq (\rfr{truth}[x][]) \cusand (\rfr{truth}[y][]) \cusand (x \rfr{\veebar} y = \begin{cases} \rfr{FALSE} & x = \rfr{FALSE} \cusand y = \rfr{FALSE} \\ \rfr{TRUE} & x = \rfr{FALSE} \cusand y = \rfr{TRUE} \\ \rfr{TRUE} & x = \rfr{TRUE} \cusand y = \rfr{FALSE} \\ \rfr{FALSE} & x = \rfr{TRUE} \cusand y = \rfr{TRUE} \end{cases})
\n % TODO SPAM THEOREMS EQUIVALENCES https://en.wikipedia.org/wiki/Principle_of_explosion#Proof 
\n \dfn{operatorIF}[\dfn{\implies}][x, y] \defeq (\rfr{truth}[x][]) \cusand (\rfr{truth}[y][]) \cusand (x \rfr{\implies} y = \begin{cases} \rfr{TRUE} & x = \rfr{FALSE} \cusand y = \rfr{FALSE} \\ \rfr{TRUE} & x = \rfr{FALSE} \cusand y = \rfr{TRUE} \\ \rfr{FALSE} & x = \rfr{TRUE} \cusand y = \rfr{FALSE} \\ \rfr{TRUE} & x = \rfr{TRUE} \cusand y = \rfr{TRUE} \end{cases})
\n \dfn{operatorOIF}[\dfn{\impliedby}][x, y] \defeq (\rfr{truth}[x][]) \cusand (\rfr{truth}[y][]) \cusand (x \rfr{\impliedby} y = \begin{cases} \rfr{TRUE} & x = \rfr{FALSE} \cusand y = \rfr{FALSE} \\ \rfr{FALSE} & x = \rfr{FALSE} \cusand y = \rfr{TRUE} \\ \rfr{TRUE} & x = \rfr{TRUE} \cusand y = \rfr{FALSE} \\ \rfr{TRUE} & x = \rfr{TRUE} \cusand y = \rfr{TRUE} \end{cases})
\n \dfn{operatorIIF}[\dfn{\iff}][x, y] \defeq (\rfr{truth}[x][]) \cusand (\rfr{truth}[y][]) \cusand (x \rfr{\iff} y = \begin{cases} \rfr{TRUE} & x = \rfr{FALSE} \cusand y = \rfr{FALSE} \\ \rfr{FALSE} & x = \rfr{FALSE} \cusand y = \rfr{TRUE} \\ \rfr{FALSE} & x = \rfr{TRUE} \cusand y = \rfr{FALSE} \\ \rfr{TRUE} & x = \rfr{TRUE} \cusand y = \rfr{TRUE} \end{cases})
\end{align}
\end{tcolorbox}

\subsection{Predicate shennanigans}
\begin{tcolorbox}[breakable, enhanced]
\begin{align}
   0
\n 0
\end{align}
\end{tcolorbox}

\section{Glossary}
\begin{multicols}{4}
\features
\end{multicols}

\bibliography{sample}
\end{document}
