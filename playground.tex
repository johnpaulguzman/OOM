% NOTE DUPLICATE 439
% Operators dump: \odot$ \quad $\oplus$ \quad $\otimes$ \quad $\ominus$ \quad $\oslash
\documentclass[a4paper]{article}

%% Language and font encodings
\usepackage[english]{babel}
\usepackage[utf8x]{inputenc}
\usepackage[T1]{fontenc}

%% Sets page size and margins
\usepackage[a4paper,top=1.25cm,bottom=1.25cm,left=1.0cm,right=1.0cm,marginparwidth=1.25cm]{geometry}

%% Useful packages
\usepackage{amsmath}
\usepackage{graphicx} 
\usepackage[colorinlistoftodos]{todonotes}
\usepackage[colorlinks=true, allcolors=blue]{hyperref}
\DeclareMathSizes{12}{30}{16}{12}
\usepackage[final]{pdfpages}
\usepackage{gensymb}
\usepackage{empheq}
\usepackage[section]{placeins}
\usepackage[most]{tcolorbox}
\usepackage{esint}
\usepackage{amssymb}
\usepackage{centernot}
\usepackage{multicol}
\usepackage{etoolbox}
\usepackage{xparse}
\usepackage{xcolor} % black, blue, brown, cyan, darkgray, gray, green, lightgray, lime, magenta, olive, orange, pink, purple, red, teal, violet, white, yellow.
\usepackage{soul}
\delimitershortfall-1sp
\usepackage{mleftright}
\mleftright
\def\ml{\mleft}
\def\mr{\mright}

\makeatletter
\newcommand{\tdb}[1]{\colorbox{lime}{$\displaystyle #1$}}
\newcommand{\defeq}{:=}
\newcommand{\uset}[1]{\{#1\}}
\newcommand{\otup}[1]{\langle#1\rangle}
\newcommand{\enlist}[2]{{#1}_{1}, {#1}_{2}, {#1}_{3}, \cdots, {#1}_{#2}}
\newcommand{\cusand}{,}
\newcommand{\cuspop}{.}
\newcommand{\eqComment}[1]{\text{  \# #1}}
\newcommand{\eqSep}{\text{ ;  }}
\newcommand{\n}{\\[1.5ex] \hline \nonumber \\[0ex]}
\newcommand{\m}{\nonumber \\}
\newcommand{\field}[1]{\textbf{\textit{#1}}}

%%%%%%%%%%%%%%%%%%
\newcommand{\dfnNEW}[2]{\textcolor{teal}{%
  \hypertarget{#1}{#2}%
    \protected@write\@mainaux{}{%
        \string\expandafter\string\gdef
          \string\csname\string\detokenize{#1}\string\endcsname{#2}%
    }%
  }}
\DeclareDocumentCommand{\dfn}{m o}{%
  \dfnNEW{#1}{\IfNoValueTF{#2}{#1}{#2}}
  }
  
\newcommand{\rfrNEW}[1]{%
  \hyperlink{#1}{\csname #1\endcsname}%
  }
\newcommand{\rfr}[1]{% maybe optional this as well
  \ifcsname#1\endcsname%
    \rfrNEW{#1}%
  \else%
    \textcolor{red}{#1^!}% undefined
  \fi%
}

%%%%%%%%%%%%%%%%%%
\newcommand\rfrlist[1]{%
    \forcsvlist{\rfrlist@item}{#1}
}
\newcommand\rfrlist@item[1]{\rfr{#1}\\}
\newcommand{\thmlink}[2]{{}_{\substack{\rfrlist{#1}}}^{\dfn{#2}} }
\makeatother

\title{Next-Next-Gen Notes \\
\large Object-Oriented Maths}
\author{JP Guzman}

\begin{document}
\maketitle
\allowdisplaybreaks


\thinmuskip=2mu % commas
\medmuskip=2mu % operators
\thickmuskip=2mu % equalities
\setlength{\belowdisplayskip}{0pt} \setlength{\belowdisplayshortskip}{0pt}
\setlength{\abovedisplayskip}{0pt} \setlength{\abovedisplayshortskip}{0pt}

% https://en.wikipedia.org/wiki/List_of_mathematical_abbreviations


$
\m \thmlink{POS-LDis}{THM-LDMr-1} ((x \rfr{OR} y) \rfr{OR} (\rfr{NOT} x \rfr{AND} \rfr{NOT} y) = ((x \rfr{OR} y) \rfr{OR} \rfr{NOT} x) \rfr{AND} ((x \rfr{OR} y) \rfr{OR} \rfr{NOT} y)) \cusand
\m \thmlink{POS-LCom, THM-LAsc}{THM-LDMr-2} (((x \rfr{OR} y) \rfr{OR} \rfr{NOT} x) \rfr{AND} ((x \rfr{OR} y) \rfr{OR} \rfr{NOT} y) = ((x \rfr{OR} \rfr{NOT} x) \rfr{OR} y) \rfr{AND} ((\rfr{NOT} y \rfr{OR} y) \rfr{OR} x)) \cusand
\m \thmlink{POS-LCmp}{THM-LDMr-3} (((x \rfr{OR} \rfr{NOT} x) \rfr{OR} y) \rfr{AND} ((\rfr{NOT} y \rfr{OR} y) \rfr{OR} x) = (\rfr{T} \rfr{OR} y) \rfr{AND} (\rfr{T} \rfr{OR} x)) \cusand
\m \thmlink{THM-LDom}{THM-LDMr-4} ((\rfr{T} \rfr{OR} y) \rfr{AND} (\rfr{T} \rfr{OR} x) = \rfr{T} \rfr{AND} \rfr{T}) \cusand
\m \thmlink{THM-LIdm}{THM-LDMr-5} (\rfr{T} \rfr{AND} \rfr{T} = \rfr{T}) \cusand
\m \thmlink{THM-LDMr-1, THM-LDMr-2, THM-LDMr-3, THM-LDMr-4, THM-LDMr-5}{THM-LDMr-6} ((x \rfr{OR} y) \rfr{OR} (\rfr{NOT} x \rfr{AND} \rfr{NOT} y) = \rfr{T}) \cuspop
\m \thmlink{THM-LDis}{THM-LDMr-7} ((x \rfr{OR} y) \rfr{AND} (\rfr{NOT} x \rfr{AND} \rfr{NOT} y) = (x \rfr{AND} \rfr{NOT} x \rfr{AND} \rfr{NOT} y) \rfr{OR} (y \rfr{AND} \rfr{NOT} x \rfr{AND} \rfr{NOT} y)) \cusand
\m \thmlink{POS-LCom, THM-LAsc}{THM-LDMr-8} ((x \rfr{AND} \rfr{NOT} x \rfr{AND} \rfr{NOT} y) \rfr{OR} (y \rfr{AND} \rfr{NOT} x \rfr{AND} \rfr{NOT} y) = ((x \rfr{AND} \rfr{NOT} x) \rfr{AND} \rfr{NOT} y) \rfr{OR} ((y \rfr{AND} \rfr{NOT} y) \rfr{AND} \rfr{NOT} x)) \cusand
\m \thmlink{POS-LCmp}{THM-LDMr-9} (((x \rfr{AND} \rfr{NOT} x) \rfr{AND} \rfr{NOT} y) \rfr{OR} ((y \rfr{AND} \rfr{NOT} y) \rfr{AND} \rfr{NOT} x) = (\rfr{F} \rfr{AND} \rfr{NOT} y) \rfr{OR} (\rfr{F} \rfr{AND} \rfr{NOT} x)) \cusand
\m \thmlink{THM-LDom}{THM-LDMr-10} ((\rfr{F} \rfr{AND} \rfr{NOT} y) \rfr{OR} (\rfr{F} \rfr{AND} \rfr{NOT} x) = \rfr{F} \rfr{OR} \rfr{F}) \cusand
\m \thmlink{THM-LIdm}{THM-LDMr-11} (\rfr{F} \rfr{OR} \rfr{F} = \rfr{F}) \cusand
\m \thmlink{THM-LDMr-7, THM-LDMr-8, THM-LDMr-9, THM-LDMr-10, THM-LDMr-11}{THM-LDMr-12} ((x \rfr{OR} y) \rfr{AND} (\rfr{NOT} x \rfr{AND} \rfr{NOT} y) = \rfr{F}) \cuspop
\m \thmlink{THM-LDMr-6, THM-LDMr-12, POS-LCmp}{THM-LDMr-13} (((x \rfr{OR} y) \rfr{OR} (\rfr{NOT} x \rfr{AND} \rfr{NOT} y) = \rfr{T} = (x \rfr{OR} y) \rfr{OR} \rfr{NOT} (x \rfr{OR} y)) \cusand ((x \rfr{OR} y) \rfr{AND} (\rfr{NOT} x \rfr{AND} \rfr{NOT} y) = \rfr{F} = (x \rfr{OR} y) \rfr{AND} \rfr{NOT} (x \rfr{OR} y))) \cusand
\m \thmlink{THM-LDMr-13, THM-LUNt}{THM-LDMr-14} (\rfr{NOT} x \rfr{AND} \rfr{NOT} y = \rfr{NOT} (x \rfr{OR} y)) \cusand
\m \thmlink{THM-LDMr-14, THM-Dual}{THM-LDMr} ((\rfr{NOT} x \rfr{AND} \rfr{NOT} y = \rfr{NOT} (x \rfr{OR} y)) \cusand (\rfr{NOT} x \rfr{OR} \rfr{NOT} y = \rfr{NOT} (x \rfr{AND} y))) \cuspop
\m \eqComment{Boolean De Morgan's Laws}
    \m \thmlink{IMPLIESWAT}{THM-CtrP-1} (x \rfr{IF} y = (\rfr{NOT} x) \rfr{OR} y) \cusand 
\m \thmlink{POS-LCom, THM-LInv}{THM-CtrP-2} ((\rfr{NOT} x) \rfr{OR} y = ((\rfr{NOT} \rfr{NOT} y) \rfr{OR} (\rfr{NOT} x))) \cusand 
\m \thmlink{IMPLIESWAT}{THM-CtrP-3} ((\rfr{NOT} \rfr{NOT} y) \rfr{OR} (\rfr{NOT} x) = (\rfr{NOT} y) \rfr{IF} (\rfr{NOT} x)) \cusand 
\m \thmlink{THM-CtrP-1, THM-CtrP-2, THM-CtrP-3}{THM-CtrP} (x \rfr{IF} y = (\rfr{NOT} y) \rfr{IF} (\rfr{NOT} x)) \cuspop
\m \eqComment{Contrapositive Law} 
% http://www.cs.um.edu.mt/gordon.pace/Teaching/DiscreteMaths/Laws.pdf
    \m \tdb{(\rfr{T} \rfr{IF} x = x)}
    \m \tdb{(\rfr{F} \rfr{IF} x = \rfr{T})}
    \m \tdb{(x \rfr{IF} \rfr{T} = \rfr{T})}
    \m \tdb{(x \rfr{IF} \rfr{F} = \rfr{NOT} x)}  
    \m \tdb{((x \rfr{OR} y) \rfr{IF} z) = (x \rfr{IF} z) \rfr{AND} (y \rfr{IF} z)}
    \m \tdb{(x \rfr{IF} (y \rfr{AND} z) = (x \rfr{IF} y) \rfr{AND} (x \rfr{IF} z))}
$

\section{Mathematical Logic}
\subsection{NaiveMaster}
\begin{tcolorbox}[breakable, enhanced]
\begin{align}
\n \dfn{setenum}[\{\}]
\n \rfr{setenum}
\n \tdb{\textbf{undefined terms}}
\m \dfn{set}, \dfn{element}, \dfn{-in}[\in], \dfn{-subset}[\subseteq], \dfn{-eq}[=], \dfn{-not}[\not] //, \dfn{-psubset}[\subset], \dfn{-union}[\cup], \dfn{-intersection}[\cap], \dfn{empty-set}[\emptyset]
\n \rfr{element} [x][] \rfr{-in} \rfr{set} [y][]
\m \eqComment{x belongs to y}
\n \rfr{set} [x][] \rfr{-subset} \rfr{set} [y][]
\m \eqComment{x is included in y}
\n \rfr{set} [x][] \rfr{-eq} \rfr{set} [y][] \defeq (\rfr{set} [x][] \rfr{-subset} \rfr{set} [y][], \rfr{set} [y][] \rfr{-subset} \rfr{set} [x][])
\m \eqComment{x is the same set as y}
\n \rfr{set} [x][] \rfr{-psubset} \rfr{set} [y][] ((= x \nsubseteq y)) \defeq \rfr{set} [x][] \subseteq \rfr{set} [y][], \rfr{set} [x][] \rfr{-not} \rfr{-eq} \rfr{set} [y][]
\m \eqComment{x is a proper subset of y}
\n \rfr{set} [x][] \rfr{-union} \rfr{set} [y][]
\m \eqComment{all elements in x or y}
\n \rfr{set} [x][] \rfr{-intersection} \rfr{set} [y][]
\m \eqComment{all elements in x and y}
\n \dfn{disjoint} [x, y][] \defeq \rfr{set} [x][] \rfr{-intersection} \rfr{set} [y][] \rfr{-eq} \rfr{empty-set}
\m \eqComment{disjoint sets do not intersect}
\n \uset{\enlist{e}{n}}
\m \eqComment{unordered set containing $\enlist{e}{n}$}
\m \uset{e_1, e_2, e_3} = \uset{e_3, e_1, e_2}
\n \otup{\enlist{e}{n}} := \text{ordered tuple containing $\enlist{e}{n}$}
\m \otup{e_1, e_2, e_3} \neq \otup{e_2, e_3, e_1}
\n X^k = \uset{\enlist{e}{n}}^k := \text{set of all ordered k-tuples from the elements of $\enlist{e}{n}$}
\m X^1 = \uset{\enlist{e}{n}}^1 = \uset{\otup{e_1}, \otup{e_2}, \otup{e_3}, \cdots, \otup{e_n}} = \uset{\enlist{e}{n}} = X
\n Y \times Z = \uset{\enlist{y}{i}} \times \uset{\enlist{z}{j}} := \text{Cartesian product}
\m := \bigcup_{a \leq i, b \leq j} (\uset{\otup{y_a, z_b}})
\n R^k_Y \subseteq Y^k := \text{k-tuple relation R on the set Y takes only tuples that satisfy some relation}
\m P_Y \subseteq Y := \text{property P of the set Y}
\n \otup{y, z} \in binaryRelation(R^2_X) = y R^2_X z
\m domain(Y), range(Z)
\m field(R) = Y \cup Z
\m \otup{a, b} \in inverse(R^{-1}) : \otup{b, a} \in R 
\m reflexive(R^2_X) : x R^2_X x
\m symmetric(R^2_X) : x R^2_X y = y R^2_X x
\m transitive(R^2_X) : x R^2_X y, y R^2_X z : x R^2_X z
\end{align}
\end{tcolorbox}

\end{document}
